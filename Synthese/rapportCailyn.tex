Dans l'ensemble le cours est cohérent : Les design patterns sont un nouveau concept pour la majorité d'entre nous et intéressant à apprendre, le projet nous permet d'étudier spec, conception, réalisation et tests avec une complexité variable et s'adapte donc bien aux différents étudiants.

Le bémol de ce cours fût son organisation, les derniers cours de design pattern ont eu lieu bien trop tard : Nos conceptions était trop avancées pour que ça soit valable de tout changer pour implémenter les design patterns qui nous étaient présentés. Je proposerais de scinder les deux parties (projet et design pattern) en deux parties bien séparées. En commençant par les cours de design pattern, puis en nous présentant le projet. Celà permetterait aux étudiants de connaitres les outils avant le projet (ou au pire très tôt dans le projet), au lieu de les apprendre à mi-chemin quand des solutions moins viables sont déjà implémentées.
