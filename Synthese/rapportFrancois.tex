Le projet Monkey Island a été un bon projet mêlant un ensemble de d’aspects variés comme la conception, la programmation, la réalisation de tests et enfin une qualité du code à respecter en fonction des métriques définies par l’équipe en début de projet.
Le fait de mélanger les options SE et LD est un bonne chose dans ce projet. Cela permet de partager les différentes connaissances acquises d’étudiants d’options différentes et ainsi de pouvoir monter en compétences.\\

Un bémol dans ce projet est l’apprentissage des différents design pattern qui nous ont été présentés alors que le projet avait déjà commencé. De plus, le support permettant de les programmer est pour ma part trop mince. J’aurai aimé un peu plus d’accompagnement dans les démarches à suivre pour réaliser chaque design pattern.