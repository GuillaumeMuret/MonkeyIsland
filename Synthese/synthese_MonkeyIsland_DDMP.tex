%-----------------------------------
% Source de Base rapport ESEO
%
% <Type de document>	\TypeDocument
% <Portée du document>	\PorteeDocument
%
%-----------------------------------

\documentclass[a4paper,11pt,titlepage]{article}

\usepackage[english,francais]{babel}
\usepackage[T1]{fontenc}
\usepackage[utf8]{inputenc}

\usepackage{xspace, graphicx}
\usepackage{amsmath}
\usepackage{amsfonts}
\usepackage{lastpage}
\usepackage{array}
\usepackage{hyperref}
\usepackage{color}
\usepackage{xcolor}
\usepackage{lmodern}

%------------------------------------

\usepackage{fancyhdr}
\newcommand{\TypeDocument}{Synthèse Projet}
\newcommand{\AuteurPrincipal}{Clément Pabst}
\newcommand{\NomGroupeDocument}{Equipe DDMP}
\newcommand{\IntituleDocument}{Module «~COOL, QL~»}
\newcommand{\PorteeDocument}{A l'attention des enseignants du module «~COOL, QL~».}
\pagestyle{fancy}

%%%%Ajouts Clément Pabst, Début :

\usepackage[shortlabels]{enumitem}
\setlist[itemize]{label=\textbullet}


\hypersetup{                    % parametrage des hyperliens
    colorlinks=true,                % colorise les liens
    breaklinks=true,                % permet les retours à la ligne pour les liens trop longs
    urlcolor= black,                 % couleur des hyperliens
    linkcolor= darkgray,                % couleur des liens internes aux documents (index, figures, tableaux, equations,...)
    citecolor= black                % couleur des liens vers les references bibliographiques
    pdfauthor={\AuteurPrincipal},     % author
    pdfcreator={\AuteurPrincipal},     % creator
    colorlinks=true,       % false: boxed links; true: colored links
    pdfsubject={Cahier de Test},   % subject of the document
    pdftitle={MonkeyIsland - Equipe DDMP},  %title of the document
    pdfproducer={}
}

%%%%Ajouts Clément Pabst,  Fin  :

\setlength{\hoffset}{-40pt}

\setlength{\topmargin}{-25pt}
\setlength{\headsep}{10pt}

\renewcommand{\headheight}{80pt}

\renewcommand{\headwidth}{450pt}
\setlength{\textwidth}{450pt}
\setlength{\textheight}{604pt}

\renewcommand{\footrulewidth}{0.1mm}

\fancyhf{}
        \fancyhead[LO]{\bf \includegraphics[width=60pt]{img/logoeseo.eps} \textsc{Angers}\\
                                Cycle Ingénieur $3^{\text{éme}}$ année\\
                                \NomGroupeDocument}
         \fancyhead[RO]{\bf {\large \IntituleDocument}\\
				\medskip
				{\large \TypeDocument}\\
				\medskip
				{\small\textit{version du \today}}}
         \fancyfoot[LO]{\sl \PorteeDocument}
         \fancyfoot[RO]{\thepage/\pageref{LastPage}}

\setcounter{tocdepth}{3}

%----------------------------------------
%       DOCUMENT
%----------------------------------------

\begin{document}

%---------------------------------------
\setlength\parindent{0pt}
\sloppy%
\renewcommand{\arraystretch}{1.5}

%---------------------------------------

\vspace{-2cm}
\begin{center}
{\Large {\textsc{\bf \IntituleDocument}}}
\vspace{0.4cm}\\
{\large\bf \NomGroupeDocument}

\rule[0.5ex]{0.52\textwidth}{0.1mm}

\vspace{0.2cm}
{\Large\bf{\TypeDocument}}
\vspace{0.2cm}\\
{\Large\bf\textit{\PorteeDocument}}
\end{center}

\vspace{-0.2cm}
\noindent\rule[0.5ex]{\textwidth}{0.1mm}
\textit{Ce document est une synthèse de notre projet «~Monkey Island~». Il est à destination des professeurs en charge du module.}\\
\rule[0.5ex]{\textwidth}{0.1mm}

\vspace{1cm}
\begin{tabular}{|p{5cm}|p{9cm}|}
\hline
\textbf{Auteurs} & \textbf{Contact} \\\hline
François de Broch d'Hotelans & francois.debrochdhotelans@reseau.eseo.fr\\\hline
Cailyn Davies & cailyn.davies@reseau.eseo.fr\\\hline
Guillaume Muret & guillaume.muret@reseau.eseo.fr\\\hline
\AuteurPrincipal & clement.pabst@reseau.eseo.fr\\\hline
\end{tabular}
\newpage

%------------------------------- 

\tableofcontents
% alternative pour réduire l'espacement entre les entrées de la table des matières
% (la valeur numérique peut être adaptée au besoin) : 
%{\setlength{\baselineskip}{0.96\baselineskip}\tableofcontents\par}
\newpage

%-------------------------------
%-----------------------------------
% Projet Systèmes Critiques
%
% <Type de document>
% <Portée du document>
% - Introduction
%-----------------------------------

\section{Introduction}
\label{sec:introduction}

\subsection{Objectif}
\label{subsec:introduction:objectif}

L'objectif de ce document est de présenter les tests réalisés dans le cadre du projet Monkey Island. On y vérifiera le bon déroulement des fonctionnalités évoquées dans le plan de test.

\subsection{Portée}
\label{subsec:introduction:portee}

Ce document est à destination des professeurs en charge de l'évaluation du module «~COOL, QL~» de l'année 2017-2018. Il n'a pour vocation qu'une exploitation pédagogique, interne à l'ESEO.



\newpage
\section{Réalisations et Validation}
\label{sec:rEtV}

Lors de ce projet, il nous était demandé de concevoir, développer, tester un programme informatique basé sur le jeux Monkey Island.\\
Ce document présente les fonctionnalités qui ont été développées, testées et validées selon un certain nombre d’incrément.

\subsection{Premier incrément}
\label{subsec:inc1}

\begin{itemize}
	\item Déplacement des singes erratiques

	Les singes erratiques se déplacent aléatoirement avec une probabilité de 25 \% d’aller dans une des quatre directions haut, bas, droite ou gauche.

	\item Gestion des pirates

	Les pirates se déplacent sur la carte dans l’objectif de trouver le trésor avant de se faire manger par l'un des différents singes du jeu, le tout sans arriver à cours d'énergie.

	\item Gestion du trésor

	La présence du trésor sur la carte du jeu. Découvrir ce trésor permet à un pirate de mettre fin à la partie, et d'en sortir victorieux.

	\item Gestion du fichier de configuration

	Un fichier .json permettant de configurer les différents paramètres et données du jeu.
\end{itemize}

\subsection{Deuxième incrément}
\label{subsec:inc2}

\begin{itemize}
	\item Communication Client – Serveur

	La communication entre le serveur du jeu et les différents clients (joueurs) est faite. Ainsi les joueurs peuvent jouer en temps réel sur la même carte de jeu.

	\item Gestion de l’énergie de chaque pirate

	L’énergie de chaque pirate, fixée à 20 (fichier de configuration), est fonctionnelle. Si un pirate n’a plus d’énergie, il meurt. De plus, s’il dépasse le seuil de son énergie maximum, il devient saoûl.

	\item Gestion des bouteilles de rhum

	Les bouteilles de rhum peuvent être :
	\begin{itemize}
		\item soit visible, alors les pirates peuvent les boire et ainsi augmenter leur énergie
		\item soit invisible, les pirates ne peuvent pas les voir ni les boire.
	\end{itemize}

	Un timer est mis en place afin de faire ré-apparaître les bouteilles de rhum.
\end{itemize}

\subsection{Troisième incrément}
\label{subsec:inc3}

\begin{itemize}
	\item Gestion des parties

	La découverte du trésor par un pirate signifie la fin de la partie en cours.

	\item Gestion des singes chasseurs

	Les singes chasseurs ont des déplacements « intelligent », c’est-à-dire qu’ils se dirigent vers le pirate le plus proche de leur position. Des tests de modèle de déplacement systématique ont été validés afin d’éviter toute faille dans la gestion des déplacements des singes chasseurs.

	\item Gestion de l’alcoolémie du pirate

	L’alcoolémie du pirate est fonctionnelle. Si un pirate dépasse le seuil de son énergie maximum (20) , il devient saoûl et se déplace comme un singe erratique avec une probabilité de mouvement de 25 \% dans chaque direction (haut, bas, droite et gauche), quelle que soit la direction demandée par le joueur.
\end{itemize}
L’utilisation des différents outils à notre disposition (Checkstyle, Findbugs, Subversion) nous a permis d’avoir un code de qualité en préconisant une performance maximale.

\subsection{Fonctions non réalisées}
\label{subsec:irrealise}
Le projet, ainsi que les trois incréments ont été testés et validés. Nous n'avons donc actuellement aucune fonctionnalité manquante.
\newpage
\section{Retours \& Critiques // Rapports d'étonnement}
\label{sec:rapportsE}

\subsection{François de Broch d’Hotelans}
\label{subsec:rapportsE:Francois}
Le projet Monkey Island a été un bon projet mêlant un ensemble de d’aspects variés comme la conception, la programmation, la réalisation de tests et enfin une qualité du code à respecter en fonction des métriques définies par l’équipe en début de projet.
Le fait de mélanger les options SE et LD est un bonne chose dans ce projet. Cela permet de partager les différentes connaissances acquises d’étudiants d’options différentes et ainsi de pouvoir monter en compétences.\\

Un bémol dans ce projet est l’apprentissage des différents design pattern qui nous ont été présentés alors que le projet avait déjà commencé. De plus, le support permettant de les programmer est pour ma part trop mince. J’aurai aimé un peu plus d’accompagnement dans les démarches à suivre pour réaliser chaque design pattern.

\subsection{Cailyn Davies}
\label{subsec:rapportsE:Cailyn}
Dans l'ensemble le cours est cohérent : Les design patterns sont un nouveau concept pour la majorité d'entre nous et intéressant à apprendre, le projet nous permet d'étudier spec, conception, réalisation et tests avec une complexité variable et s'adapte donc bien aux différents étudiants.

Le bémol de ce cours fût son organisation, les derniers cours de design pattern ont eu lieu bien trop tard : Nos conceptions était trop avancées pour que ça soit valable de tout changer pour implémenter les design patterns qui nous étaient présentés. Je proposerais de scinder les deux parties (projet et design pattern) en deux parties bien séparées. En commençant par les cours de design pattern, puis en nous présentant le projet. Celà permetterait aux étudiants de connaitres les outils avant le projet (ou au pire très tôt dans le projet), au lieu de les apprendre à mi-chemin quand des solutions moins viables sont déjà implémentées.


\subsection{Guillaume Muret}
\label{subsec:rapportsE:Guillaume}
Ce projet a été une excellente expérience de groupe rassemblant différentes personnalités et différents niveaux et spécialités dans ce projet informatique. 
La réalisation d'une conception et d'un code en respectant une qualité stricte et en implémentant les tests associés à notre travail est très enrichissante et montre la rigueure que doit suivre un ingénieur tout au long d'un projet.
L'apprentissage des patterns Java est essentiel dans la programmation orienté objet et nous a permis d'évoluer considérablement notre façon de concevoir et de coder en orienté objet.
Cependant, il aurait été plus intéressant d'apprendre les patterns avant le début de notre projet car celui-ci aurait été optimisé si nous avions vu l'ensemble des patterns avant son commencement. 
Dans l'ensemble, ce projet a été à mon sens une réussite car nous avons rempli l'ensemble des objectifs à accomplir et nous avons appris et tiré beaucoup de leçon.

\subsection{Clément Pabst}
\label{subsec:rapportsE:Clement}
\input{rapportClement}
\end{document}

%--- END