%--------------------------------------
% III Périmètre de test
%--------------------------------------

\chapter{Périmètre de test}
	\section{Composants}
		\subsection{Composants concernés par les tests}
		
Seront concernés par l'activité de test les composants logiciels développés durant le projet : 
\begin{itemize}
\item[-]\textbf{Serveur} : le logiciel serveur.
\end{itemize}


		\subsection{Composants non concernés par les tests}
		
Les supports d'exécutions matériels ne seront pas concernés par les tests.

Les supports d'exécutions logiciels suivant ne seront pas concernés par les tests:
\begin{itemize}
\item[-]\textbf{Le client}, en effet, seul le serveur est testé.\\
\end{itemize}

Les supports de communication matériels suivant ne seront pas concernés par les tests: 
\begin{itemize}
\item[-]\textbf{Le réseau de l'école.}\\
\end{itemize}

Les supports de communication logiciels: 
\begin{itemize}
\item[-]\textbf{la pile TCP/IP.}
\end{itemize}
\newpage
		
	\section{Fonctionnalités}
	\label{refpFonctionnalités}

	\subsection{Incrément 1}
\begin{table}[!h]
\begin{center}
\begin{tabular}{|p{10cm}|c|c|}
\hline
\textbf{Fonctionnalités à tester} & \textbf{Testé} & \textbf{Validation} \\\hline
Déplacement des singes erratiques			& oui & A \\\hline
Gestion des pirates (déplacement et décès)	& oui & A \\\hline
%Gestion du fichier de configuration & TODO & - \\\hline
\end{tabular}
\end{center}
\caption{Table des fonctionnalités testées pour l'incrément 1}
\end{table}

\begin{table}[!h]
\begin{center}
\begin{tabular}{|p{10cm}|c|c|}
\hline
\textbf{Déplacement des singes erratiques} & \textbf{Testé} & \textbf{Validation} \\
\hline
Déplacement ok terre vide H-B-G-D			& oui & A\\\hline
Pas de déplacement en diagonale terre vide	& oui & A\\\hline
Obligation de déplacement terre vide		& oui & A\\\hline
Pas de déplacement possible (mer)			& oui & A\\\hline
Pas de déplacement possible (erratiques)	& oui & A\\\hline
Pas de déplacement possible (chasseurs)		& oui & A\\\hline
Déplacement ok - trésor caché -> caché		& oui & A\\\hline
Déplacement ok - pirate vivant -> mort		& oui & A\\\hline
Déplacement ok - pirate mort -> mort		& oui & A\\\hline
Déplacement ok - rhum dispo -> dispo		& oui & A\\\hline
Equiprobabilité pour 4						& oui & A\\\hline
Equiprobabilité pour 3						& oui & A\\\hline
Equiprobabilité pour 2						& oui & A\\\hline
Non équiprobabilité stricte pour 4			& oui & A\\\hline
Non équiprobabilité stricte pour 3			& oui & A\\\hline
Non équiprobabilité stricte pour 2			& oui & A\\\hline
Schéma non répétitif pour 4					& oui & B+\\\hline
Schéma non répétitif pour 3					& oui & B+\\\hline
Schéma non répétitif pour 2					& oui & B+\\\hline
Pas de déplacement "nuageux"				& oui & B+\\\hline
\end{tabular}
\end{center}
\caption{Table des tests liés au déplacement des singes erratiques}
\end{table}

\begin{table}[!h]
\begin{center}
\begin{tabular}{|p{10cm}|c|c|}
\hline
\textbf{Gestion des pirates (déplacement et décès) } & \textbf{Testé} & \textbf{Validation} \\\hline
Déplacement ok terre vide H-B-G-D				& oui & A \\\hline
Pas de déplacement en diagonale terre vide		& oui & A \\\hline
Pas de déplacement si instruction = 0-0			& oui & A \\\hline
Pas de déplacement possible (mer)				& oui & A \\\hline
Pas de déplacement possible (pirate)			& oui & A \\\hline
Pas de déplacement possible (vide)				& oui & A \\\hline
Pas de déplacement hors carte					& oui & A \\\hline
Déplacement sur singe erratique -> mort			& oui & A \\\hline
Déplacement sur singe chasseur -> mort			& oui & A \\\hline
Pas de déplacement si mort, case terre			& oui & A \\\hline
Pas de déplacement si mort, case terre et rhum	& oui & A \\\hline
Pas de résurrection d'un pirate mort lors de réapparition de rhum	& oui & A \\\hline
Déplacement sur trésor -> découverte trésor et fin de partie		& oui & A \\\hline
Déplacement sur trésor et singe -> découverte trésor, fin de partie et mort & oui & A \\\hline

\end{tabular}
\end{center}
\caption{Table des tests liés à la gestion des pirates (déplacement et décès) }
\end{table}

\newpage
	\subsection{Incrément 2}
\begin{table}[!h]
\begin{center}
\begin{tabular}{|p{10cm}|c|c|}
\hline
\textbf{Fonctionnalités à tester} & \textbf{Testé} & \textbf{Validation} \\
\hline
Communication client-serveur & oui & A \\\hline
Gestion de l’énergie de chaque pirate & oui & A \\\hline
\end{tabular}
\end{center}
\caption{Table des fonctionnalités testées pour l'incrément 2}
\end{table}

\begin{table}[!h]
\begin{center}
\begin{tabular}{|p{10cm}|c|c|}
\hline
\textbf{Communication client-serveur} & \textbf{Testé} & \textbf{Validation} \\\hline
Inscription du pirate							& oui & A \\\hline
Déplacements HBGD (4tests)						& oui & A \\\hline
NON-Déplacement de 2 cases HBGD (4tests)		& oui & A \\\hline
NON-Déplacement en diagonale (4tests)			& oui & A \\\hline
NON-Déplacement de 2 cases en diagonale (4tests) & oui & A \\\hline
\end{tabular}
\end{center}
\caption{Table des tests liés à la communication client-serveur}
\end{table}

\begin{table}[!h]
\begin{center}
\begin{tabular}{|p{10cm}|c|c|}
\hline
\textbf{Gestion de l’énergie de chaque pirate} & \textbf{Testé} & \textbf{Validation} \\\hline
Mort sans énergie & oui & A \\\hline
Déplacement sur rhum -> énergie<ENERGIE\_MAX	& oui & A \\\hline
Déplacement sur rhum -> énergie>ENERGIE\_MAX	& oui & A \\\hline
Déplacement sur rhum -> énergie=ENERGIE\_MAX	& oui & A \\\hline
\end{tabular}
\end{center}
\caption{Table des tests liés à la gestion de l’énergie de chaque pirate}
\end{table}

\newpage
	\subsection{Incrément 3}
\begin{table}[!h]
\begin{center}		
\begin{tabular}{|p{10cm}|c|c|}%{|p{6,3cm}|p{2,7cm}|p{4cm}|}
\hline
\textbf{Fonctionnalités à tester} & \textbf{Testé} & \textbf{Validation} \\\hline
Gestion des parties					& oui & A \\\hline
Gestion des singes chasseurs		& oui & A \\\hline
Gestion de l’alcoolémie du pirate	& oui & A \\\hline
\end{tabular}
\end{center}
\caption{Table des fonctionnalités testées pour l'incrément 3}
\end{table}

\begin{table}[!h]
\begin{center}
\begin{tabular}{|p{10cm}|c|c|}
\hline
\textbf{Gestion des parties} & \textbf{Testé} & \textbf{Validation} \\\hline
Partie finie -> redémarre si le pirate va sur le trésor & oui & A \\\hline
Partie finie si plus de pirate sur la carte & oui(visuel) & A \\\hline
Partie finie -> redémarre si plus de pirate vivant sur la carte & oui(visuel) & A \\\hline
\end{tabular}
\end{center}
\caption{Table des tests liés à la gestion des parties}
\end{table}

\begin{table}[!h]
\begin{center}
\begin{tabular}{|p{10cm}|c|c|}
\hline
\textbf{Gestion des singes chasseurs} & \textbf{Testé} & \textbf{Validation} \\\hline
Déplacement ok terre vide H-B-G-D 			& oui & A \\\hline
Pas de déplacement en diagonale terre vide 	& oui & A \\\hline
Obligation de déplacement terre vide 		& oui & A \\\hline
Pas de déplacement possible (mer) 			& oui & A \\\hline
Pas de déplacement possible (erratiques)	& oui & A \\\hline
Pas de déplacement possible (chasseurs)		& oui & A \\\hline
Déplacement ok - trésor caché -> caché		& oui & A \\\hline
Déplacement ok - rhum dispo -> dispo		& oui & A \\\hline
Déplacement suis le pirate					& oui & A \\\hline
Déplacement ok - pirate vivant -> mort		& oui & A \\\hline
Déplacement ok - pirate mort -> mort		& oui & A \\\hline
\end{tabular}
\end{center}
\caption{Table des tests liés à la gestion des singes chasseurs}
\end{table}

\begin{table}[!h]
\begin{center}
\begin{tabular}{|p{10cm}|c|c|}
\hline
\textbf{Gestion de l’alcoolémie du pirate} & \textbf{Testé} & \textbf{Validation} \\\hline
Sobre sur rhum (visible) -> énergie>MAX -> en ébriété				& oui & A \\\hline
Sobre sur rhum (invisible) -> énergieNoChange -> PAS en ébriété		& oui & A \\\hline
En ébriété sur rhum (visible) -> en ébriété							& oui & A \\\hline
En ébriété sur rhum (invisible) -> énergie>MAX -> en ébriété		& oui & A \\\hline
En ébriété sur rhum (invisible) -> énergie=MAX -> PAS en ébriété	& oui & A \\\hline
Pirate sur rhum invisible -> visible -> énergie++					& oui & A \\\hline
En ébriété : Probabilité < 1 d'aller là où le joueur demande			& oui & B+ \\\hline
\end{tabular}
\end{center}
\caption{Table des tests liés à la gestion de l’alcoolémie du pirate}
\end{table}

\newpage
		
	\section{Critères d'acceptation des tests}
	
Pour juger de la performance d’un test nous utiliserons le système de notation suivant :

\begin{table}[!h]
\begin{center}	
\begin{tabular}{|p{2cm}|p{9cm}|}
\hline
\textbf{Note} & \textbf{Résultats pour des tests nominaux}\\
\hline
A & Entre 80\% et 100\% de réussite\\
\hline
B & Entre 50\% et 80\%(exclus) de réussite\\
\hline
C & Moins de 50\% de réussite\\
\hline
\end{tabular}
\end{center}
\caption{Table des critères d'acceptation des tests Nominaux}
\end{table}