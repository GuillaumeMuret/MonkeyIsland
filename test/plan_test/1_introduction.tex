%--------------------------------------
%I Introduction
%--------------------------------------

\chapter{Introduction}

	\section{Objet}

Ce document décrit l'activité de test qui sera menée par MonkeyIsland\_DDMP  durant le « COOL QT » dans le but de valider le produit « Monkey Island » (MoI). Il est rédigé sous la responsabilité de Clément Pabst et sous la direction du Chef de Projet : Guillaume Muret.\\

	\section{Portée}

Sont concernés par ce document :
\begin{itemize}
\item[-] \textbf{les testeurs} : afin que ceux-ci sachent ce qu'ils vont tester, comment ils le testent et comment ils rendent compte des résultats de ces tests ;
\item[-] \textbf{les développeurs} : à titre informatif, afin que ceux-ci sachent comment va être validée leur production ; à titre indicatif afin qu'ils sachent, par la description de la gestion des anomalies, comment ils s'interfaceront avec l'équipe de test ;
\item[-] \textbf{les professeurs} : ce plan de test, ainsi que son implication, feront l'objet d'audits par le corps professoral.\\
 \end{itemize}
 
\newpage 
 
	\section{Copyright}

Le présent document est la propriété de MonkeyIsland\_DDMP. Il est diffusé pour les seuls besoins du projet concerné. Il ne doit pas être reproduit, entièrement ou partiellement, ou employé pour tout autre but sans autorisation écrite préalable de Clément Pabst ou Guillaume Muret, et à la condition que cette notification soit incluse dans une telle reproduction. Aucune information quant au contenu ou aux thèmes de ce document ne peut être communiquée de quelque façon à un tiers sans autorisation écrite de Clément Pabst ou Guillaume Muret.
L’équipe professorale de l'ESEO en charge du projet Monkey Island 2017/2018 échappe à ces restrictions et peut utiliser le présent document pour toute utilisation qui leur convient dans le contexte du projet Monkey Island, et ce sans autorisation.

	\section{Termes et abréviations}

Voici les termes et abréviations nécessaires à la compréhension de l'activité de test :\\

\begin{table}[!h]
\begin{center}
\begin{tabular}{|p{5cm}|p{9cm}|}
\hline
\textbf{Terme ou abréviation} & \textbf{Signification}\\
\hline
TU (Test unitaire) & Test sur une partie précise du programme visant à contrôler la fiabilité des unités logicielles développées, vérifier le respect de leurs conceptions et à identifier les erreurs logiques.\\
\hline
TI (Test d’intégration) & Test sur un ensemble de fonctions visant à démontrer la stabilité et la cohérence des interfaces et des interactions des unités logicielles et à vérifier le respect de la conception générale.\\
\hline
TV (Test de validation) & Test visant à valider l’adéquation aux spécifications logicielles\\
\hline
DT (Données de test) & Entrées du programme à tester\\
\hline
TC (Cas de test) & Regroupe deux séquences d’action : une pour une pré-condition et une autre pour utiliser les données de test d’un jeu de test\\
\hline
JT (Jeux de test) & Ensemble des données de test\\
\hline
Valider & S’assurer que le logiciel ait les fonctions attendues\\
\hline
Vérifier & S’assurer que le logiciel fonctionne correctement\\
\hline
Test nominal & Test avec des données d’entrée valides\\
\hline
Test de robustesse & Test avec des données d’entrée invalides\\
\hline
L'école & L'ESEO\\
\hline
\end{tabular}
\end{center}
\caption{Table des termes et abréviations}
\end{table}