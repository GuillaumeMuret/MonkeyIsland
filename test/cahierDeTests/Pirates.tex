%-----------------------------------
% Projet Systèmes Critiques
%
% <Type de document>
% <Portée du document>
% - <titre section>
%-----------------------------------

\section{Pirates}
\label{sec:pirates}

\subsection{Déplacements}
\label{subsec:piratesDeplacements}

Le but ici va être de valider l'ensemble des déplacements des pirates, incluant leurs interactions avec les autres éléments du jeu (autres personnages, objets et fin de partie).\\
Au cours de ses tests, on peut tester visuellement que les singes respectent le cahier des charges.

\subsubsection{Déplacements de base d'un Pirate}\label{subsubsec:depBasePirate}
\textbf{Résumé :} S'assurer que l'ensemble des déplacements d'un pirate fonctionne. Tous les déplacements sont faits sans détours.\\
\textbf{Précondition :} Pirate placé, entouré de cases terre vides. Une case mer est présente plus loin ainsi qu'un autre pirate qui restera immobile. Le trésor est en place lui aussi, à un endroit connu du testeur. L'ensemble des déplacements nécessaire au scénario est inférieur à l'énergie de départ du pirate.
\begin{table}[!h]
	\begin{center}
		\begin{tabular}{|p{6pt}|c|c|}%p{10cm}
			\hline
			& \textbf{Etape} & \textbf{Résultat(s) Attendu(s)} \\\hline
			1 & Presser successivement H-B-G-D & Déplacement du pirate H-B-G-D  \\\hline
			2 & Presser simultanément HD||DB||BG||GH & Pas de déplacement \\\hline
			3 & Déplacer le pirate sur la case mer & Pas de déplacement \\\hline
			4 & Déplacer le pirate sur l'autre pirate & Pas de déplacement \\\hline
			5 & Déplacer le pirate en dehors de la map & Pas de déplacement \\\hline
			6 & Déplacer le pirate sur le trésor & Victoire et redémarrage de la partie \\\hline
		\end{tabular}
	\end{center}
	\caption{Scénario de déplacement d'un Pirate, obstacle et victoire}
\end{table}

\newpage
\subsubsection{Déplacements \& Interactions d'un Pirate avec d'autres éléments du jeu}
\textbf{Résumé :} S'assurer que la rencontre avec un singe tue le pirate.\\
\textbf{Précondition :} Pirate placé, entouré de cases terre vides. Une bouteille de rhum est placée, visible et pleine, juste à côté du pirate. Un singe chasseur et un singe erratique sont présents sur la carte.\\
\begin{table}[!h]
	\begin{center}
		\begin{tabular}{|p{6pt}|c|c|}
			\hline
			& \textbf{Etape} & \textbf{Résultat(s) Attendu(s)} \\\hline
			1 & Déplacer le pirate sur & Mort du pirate et\\
			 & le singe erratique/chasseur & redémarrage de la partie \\\hline
		\end{tabular}
	\end{center}
	\caption{Scénario de déplacement d'un Pirate, rencontre avec un singe erratique/chasseur}
\end{table}

\textbf{Précondition supplémentaire :} Un autre pirate est présent et se contente d'éviter les singes sans autre interraction.
\begin{table}[!h]
	\begin{center}
		\begin{tabular}{|p{6pt}|c|c|}
			\hline
			& \textbf{Etape} & \textbf{Résultat(s) Attendu(s)} \\\hline
			1 & Déplacer le pirate sur le rhum & Augmentation de l'énergie et\\
			 &  & changement d'état (DRUNK) \\\hline
			2 & Attendre un singe (rester statique) & Mort du pirate \\\hline
			3 & Attente de la réapparition du rhum & Réapparition du rhum, pas de\\
			 &  & changement de l'état(DEAD) du pirate \\\hline
			4 & Déplacer le pirate & Pas de déplacement \\\hline
			5 & Le second pirate fini par s'arrêter & Le singe chasseur tue le pirate \\
			 & & et la partie redémarre. \\\hline
		\end{tabular}
	\end{center}
	\caption{Scénario de déplacement d'un Pirate, rencontre avec un singe et mort sur rhum}
\end{table}

\begin{table}[!h]
	\begin{center}
		\begin{tabular}{|p{6pt}|c|c|}%p{10cm}
			\hline
			& \textbf{Etape} & \textbf{Résultat(s) Attendu(s)} \\\hline
			1 & Déplacer le pirate à côté du trésor & Rien \\
			 & sans autre interraction &  \\\hline
			 & Attendre que le singe chasseur soit sur & Découverte du trésor,\\
			2 & la case du trésor puis déplacer le pirate &   fin de partie\\
			 & sur la case contenant le trésor et le singe &  et mort du pirate\\\hline
		\end{tabular}
	\end{center}
	\caption{Scénario de déplacement d'un Pirate, fin de partie et mort simultanées}
\end{table}

\newpage
\subsection{Energie}
\label{subsec:piratesEnergie}
\textbf{Résumé :} Le but ici va être de valider les conséquences d'un manque d'énergie du pirate ainsi que ses rencontres avec du rhum selon son niveau d'énergie.\\
\textbf{Précondition :} Pirate placé, une carte peuplée uniquement de cases terre vides ainsi que de trois bouteilles de rhum.
\begin{table}[!h]
	\begin{center}
		\begin{tabular}{|p{6pt}|c|c|}%p{10cm}
			\hline
			& \textbf{Etape} & \textbf{Résultat(s) Attendu(s)} \\\hline
			1 & Déplacer le pirate sur une bouteille avec & Déplacements suivants normaux \\
			 & l'énergie<ENERGIE MAX après recharge & \\\hline

			2 & Déplacer le pirate sur une bouteille avec & Déplacements suivants normaux \\
			 & l'énergie=ENERGIE MAX après recharge & \\\hline
			3 & Déplacer le pirate sur une bouteille avec & Déplacements suivants aléatoires jusqu'à une \\
			 & l'énergie>ENERGIE MAX après recharge & baisse d'énergie => énergie=ENERGIE MAX\\\hline
			4 & Déplacer le pirate jusqu'à énergie=0 & Mort du pirate \\\hline
		\end{tabular}
	\end{center}
	\caption{Scénario de déplacement d'un Pirate, ainsi que son rapport à l'énergie}
\end{table}