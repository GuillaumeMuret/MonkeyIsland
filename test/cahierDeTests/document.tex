%-----------------------------------
% Source de Base rapport ESEO
%
% <Type de document>	\TypeDocument
% <Portée du document>	\PorteeDocument
%
%-----------------------------------

\documentclass[a4paper,11pt,titlepage]{article}

\usepackage[english,francais]{babel}
\usepackage[T1]{fontenc}
\usepackage[utf8]{inputenc}

\usepackage{xspace, graphicx}
\usepackage{amsmath}
\usepackage{amsfonts}
\usepackage{lastpage}
\usepackage{array}
\usepackage{hyperref}

%------------------------------------

\usepackage{fancyhdr}
\newcommand{\TypeDocument}{Cahier de Tests}
\newcommand{\AuteurPrincipal}{Clément Pabst}
\newcommand{\NomGroupeDocument}{Equipe DDMP}
\newcommand{\IntituleDocument}{Module «~COOL, QT~»}
\newcommand{\PorteeDocument}{A l'attention des enseignants du module «~COOL, QT~» 2017/2018.}
\pagestyle{fancy}

%%%%Ajouts Clément Pabst, Début :

\renewcommand{\labelitemi}{-}

\hypersetup{
%bookmarks=true,         % show bookmarks bar?
%unicode=false,          % non-Latin characters in Acrobat’s bookmarks
%pdftoolbar=true,        % show Acrobat’s toolbar?
%pdfmenubar=true,        % show Acrobat’s menu?
%pdffitwindow=true,      % page fit to window when opened
pdftitle={Cahier de Tests - Equipe DDMP},    % title
pdfauthor={\AuteurPrincipal},     % author
pdfcreator={\AuteurPrincipal},     % creator
%pdfsubject={Mon document LaTeX},   % subject of the document
%pdfnewwindow=true,      % links in new window
%pdfkeywords={blablaa, blablab}, % list of keywords
colorlinks=true,       % false: boxed links; true: colored links
linkcolor=green,          % color of internal links
%citecolor=green,        % color of links to bibliography
%filecolor=magenta,      % color of file links
%urlcolor=cyan           % color of external links
}

%%%%Ajouts Clément Pabst,  Fin  :

\setlength{\hoffset}{-40pt}

\setlength{\topmargin}{-25pt}
\setlength{\headsep}{10pt}

\renewcommand{\headheight}{80pt}

\renewcommand{\headwidth}{450pt}
\setlength{\textwidth}{450pt}
\setlength{\textheight}{604pt}

\renewcommand{\footrulewidth}{0.1mm}

\fancyhf{}
        \fancyhead[LO]{\bf \includegraphics[width=60pt]{img/logoeseo.eps} \textsc{Angers}\\
                                Cycle Ingénieur $3^{\text{éme}}$ année\\
                                \NomGroupeDocument}
         \fancyhead[RO]{\bf {\large \IntituleDocument}\\
				\medskip
				{\large \TypeDocument}\\
				\medskip
				{\small\textit{version du \today}}}
         \fancyfoot[LO]{\sl \PorteeDocument}
         \fancyfoot[RO]{\thepage/\pageref{LastPage}}

\setcounter{tocdepth}{3}

%----------------------------------------
%       DOCUMENT
%----------------------------------------

\begin{document}

%---------------------------------------

\sloppy%
\renewcommand{\arraystretch}{1.5}

%---------------------------------------

\vspace{-2cm}%
\begin{center}%
{\Large {\textsc{\bf \IntituleDocument}}}
\vspace{0.4cm}\\
{\large\bf \NomGroupeDocument}

\rule[0.5ex]{0.52\textwidth}{0.1mm}

\vspace{0.2cm}
{\Large\bf{\TypeDocument}}
\vspace{0.2cm}\\
{\Large\bf\textit{\PorteeDocument}}
\end{center}

\vspace{-0.2cm}
\noindent\rule[0.5ex]{\textwidth}{0.1mm}
\textit{Ce document est un cahier de tests dédié au projet «~Monkey Island~». Il est à destination des professeurs en charge de l'évaluation du module de l'année 2017-2018.}\\
\rule[0.5ex]{\textwidth}{0.1mm}

\vspace{1cm}
\begin{tabular}{|p{5cm}|p{9cm}|}
\hline
\textbf{Auteur} & \textbf{Contact} \\\hline
\AuteurPrincipal & clement.pabst@reseau.eseo.fr\\\hline
\end{tabular}

\vspace{0.5cm}
\begin{tabular}{|p{5cm}|p{9cm}|}\hline
\textbf{Relecteur} & \textbf{Contact} \\\hline
François de Broch d'Hotelans & francois.debrochdhotelans@reseau.eseo.fr\\\hline
Cailyn Davies & cailyn.davies@reseau.eseo.fr\\\hline
Guillaume Muret & guillaume.muret@reseau.eseo.fr\\\hline
\end{tabular}

%\vspace{1cm}
%\begin{tabular}{|p{1.5cm}|p{2cm}|p{10.1cm}|}\hline
%\textbf{Version} & \textbf{Date} & \textbf{Commentaire} \\\hline
%<x.y> & <date> & Ajout<s> : <sections et/ou sous-sections ajoutées> \newline Suppression<s> : <sections et/ou sous-sections supprimées> \newline Modification<s> : <sections et/ou sous-sections modifiées> \newline Mise<s> à jour : <sections et/ou sous-sections mises à jour> \\
%<...> & <...> & <...> \\
%<1\iere{}> & <date> & <sections présentes> \\\hline
%\end{tabular}
\newpage

%------------------------------- 

\tableofcontents
% alternative pour réduire l'espacement entre les entrées de la table des matières
% (la valeur numérique peut être adaptée au besoin) : 
%{\setlength{\baselineskip}{0.96\baselineskip}\tableofcontents\par}
\newpage

%-------------------------------

%-----------------------------------
% Projet Systèmes Critiques
%
% <Type de document>
% <Portée du document>
% - Introduction
%-----------------------------------

\section{Introduction}
\label{sec:introduction}

\subsection{Objectif}
\label{subsec:introduction:objectif}

L'objectif de ce document est de présenter les tests réalisés dans le cadre du projet Monkey Island. On y vérifiera le bon déroulement des fonctionnalités évoquées dans le plan de test.

\subsection{Portée}
\label{subsec:introduction:portee}

Ce document est à destination des professeurs en charge de l'évaluation du module «~COOL, QL~» de l'année 2017-2018. Il n'a pour vocation qu'une exploitation pédagogique, interne à l'ESEO.



\newpage
%
%%-----------------------------------
% Projet Systèmes Critiques
%
% <Type de document>
% <Portée du document>
% - Références
%-----------------------------------

\section{Références}
\label{sec:references}

\subsection{Normes et standards}
\label{subsec:reference:normes}

\begin{tabular}{|l|l|l|}
\hline
\textbf{Norme ou standard} & \textbf{Version} & \textbf{Source} \\
\hline
<nom du document> & <version> & <auteur(s) ou organisme(s)>\\
\hline
\end{tabular} 


\subsection{Documents externes}
\label{subsec:references:externes}


\begin{tabular}{|l|l|l|}
\hline
\textbf{Document} & \textbf{Version} & \textbf{Source} \\
\hline
<nom du document> & <version> & <auteur(s) ou organisme(s)>\\
\hline
\end{tabular} 


\subsection{Documents internes}
\label{subsec:references:internes}


\begin{tabular}{|l|l|l|l|}
\hline
\textbf{Document} & \textbf{Version} & \textbf{Source} \\
\hline
<nom du document> & <version> & <localisation du document>\\
\hline
\end{tabular} 



%\newpage

% \input{Section1}
% \newpage
% 
% \input{Section2}
% \newpage
% 
%-----------------------------------
% Projet Systèmes Critiques
%
% <Type de document>
% <Portée du document>
% - <titre section>
%-----------------------------------

\section{Incrément 1}
\label{sec:increment1}

Introduction sur la section.

\subsection{Déplacement des singes erratiques}
\label{subsec:titresection:titresoussection1}

Déplacement ok terre vide H-B-G-D

\begin{tabular}{|p{1.5cm}|p{2cm}|p{10.1cm}|}\hline
\textbf{Version} & \textbf{Date} & \textbf{Commentaire} \\\hline
<x.y> & <date> & Ajout<s> : <sections et/ou sous-sections ajoutées> \newline Suppression<s> : <sections et/ou sous-sections supprimées> \newline Modification<s> : <sections et/ou sous-sections modifiées> \newline Mise<s> à jour : <sections et/ou sous-sections mises à jour> \\
<...> & <...> & <...> \\
<1\iere{}> & <date> & <sections présentes> \\\hline
\end{tabular}


Pas de déplacement en diagonale terre vide 
Obligation de déplacement terre vide
Pas de déplacement possible (mer)
Pas de déplacement possible (erratiques)
Pas de déplacement possible (chasseurs)
Déplacement ok - trésor caché -> caché
Déplacement ok - pirate vivant -> mort
Déplacement ok - pirate mort -> mort
Déplacement ok - rhum dispo -> dispo
Equiprobabilité pour 4
Equiprobabilité pour 3
Equiprobabilité pour 2
Non équiprobabilité stricte pour 4
Non équiprobabilité stricte pour 3
Non équiprobabilité stricte pour 2
Schéma non répétitif pour 4
Schéma non répétitif pour 3
Schéma non répétitif pour 2
Pas de déplacement ”nuageux”


\subsection{Gestion des pirates}
\label{subsec:titresection:titresoussection2}

Déplacement ok terre vide H-B-G-D
Pas de déplacement en diagonale terre vide
Pas de déplacement si instruction = 0-0
Pas de déplacement possible (mer)
Pas de déplacement possible (pirate)
Pas de déplacement possible (vide)
Pas de déplacement hors carte
Déplacement sur singe erratique -> mort
Déplacement sur singe chasseur -> mort
Pas de déplacement si mort, case terre
Pas de déplacement si mort, case terre et rhum
Pas de résurrection d’un pirate mort lors de réapparition de
rhum
Déplacement sur trésor -> découverte trésor et fin de partie
Déplacement sur trésor et singe -> découverte trésor, fin de
partie et mort

\newpage

\input{IncrementDeux}
\newpage

\input{IncrementTrois}
\newpage

%-----------------------------------
% Projet Systèmes Critiques
%
% <Type de document>
% <Portée du document>
% - <titre section>
%-----------------------------------

\section{Titre section}
\label{sec:titresection}

Introduction sur la section.

\subsection{Titre sous section 1}
\label{subsec:titresection:titresoussection1}

Corps de la sous-section 1.


\subsection{Titre sous section 2}
\label{subsec:titresection:titresoussection2}

Corps de la sous-section 2.


\newpage

%-----------------------------------
% Projet Systèmes Critiques
%
% <Type de document>
% <Portée du document>
% - Définitions
%-----------------------------------

\section{Définitions des termes et acronymes}
\label{sec:definitionsacronymes}

\subsection{Définitions}
\label{subsec:definitionsacronymes:definitions}



\subsection{Acronymes}
\label{subsec:definitionsacronymes:acronymes}





\newpage


\end{document}

%--- END 

